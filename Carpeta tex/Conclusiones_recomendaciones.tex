\section{Conclusiones y recomendaciones}
Para concluir,
\begin{itemize}
    \item Los puertos son de suma importancia, ya que permiten la comunicación entre otros dispositivos.
    \item Lo anterior permitió realizar el trabajo de una manera más sencilla porque las conexiones de los componentes electrónicos al mcu, así como el uso de los periféricos dieron sentido a lo que se estaba programando, desde lo más básico como la declaración de una pin como entrada o salida, como otras tareas, que suenan sencillas pero complejas si se desconoce el protocolo de comunicación de la pantalla PDC8544. 
    \item El uso de funciones ayudó a que el código se viera más ordenado, por lo que en el \texttt{loop} principal es más claro lo que está realizando. Esto es muy importante porque permitió tener un funcionamiento claro y correcto de la función del voltímetro, ya que desde las magnitudes (con cierto grado de error) asignadas en las fuentes de tensión, la normalización y el escalamiento de estos valores se mostraron correctamente en la pantalla PCD8544, y cuando se irrespetaban los umbrales de [-20,20]DC y [-14.14,14.14]AC los LEDs se encendieron, y se apagan cuando se respetaba este rango. Ahora, con respecto a la comunicación serial del arduino con la PC fue éxitosa, porque se estableció la correcta comunicación ya que lo mostrado en la pantalla se fue reflejando por consola y se guardó un archivo .csv en el directorio del proyecto. Por lo tanto, el proyecto cumple todos los requisitos solicitados.
\end{itemize}
Como recomendación se debe entender lo básico de como funciona la librería que se vaya a utilizar para poder realizar la comunicación y estar haciendo pruebas constantemente para verificar de que se esté trabajando de la forma que se espera, esto fue notable en la parte de generación del csv al igual que observar la comunicación dentro del mismo simulador porque hubieron momentos donde los datos se estaban imprimiendo de forma incorrecta.