\section{Desarrollo/Análisis}
El diagrama de bloques generales para el funcionamiento del voltímetro es el siguiente.
\input{sch2.tex}
Una vez que se activen alguna de las fuentes, en la pantalla PCD8544 se mostrará las magnitudes (despreciables, \SI{0.05}{\volt}) por default en cada canal, y al estar el botón de modo AC/DC en bajo por tanto, en esta pantalla se mostrarán los valores en DC, si éste se pone en alto entonces en la pantalla se mostrarán los valores rms en AC, ahora, dependiendo de las magnitudes que se ajusten en ambos modos los LEDs de alarma se encenderán o no, que al mismo tiempo se verán en la pantalla. Por último, al presionar el botón de comunicación serial, se establece la comunicación adecuada para generar el informe del archivo csv.\par
Luego, para realizar las medidas adecuadas se tuvo que realizar una función capaz de normalizar y escalar en el intervalo de rango solicitado: [-24, 24]. 
\input{sch3.tex}
Esta función se compone de un ciclo \texttt{for} que va iterando los valores que poseen los 4 canales y realiza una resta de 511 (la mitad de 1023) y voltaje medido, luego se hace la escala con el producto de esta substacción y la división de 48 entre 1023, todo esto para lograr manipular voltajes de \SI{0}{\volt}-\SI{5}{\volt} con el objetivo de normalizar y escalar el valor máximo medido.\par
La función para generar las alarmas tanto en DC como en AC son muy similares, por lo que el siguiente diagrama de bloques es equivalente para ambos.
\input{sch4.tex}
% AGREGAR ACÁ EL DIAGRAMA DE FLUJO PARA LA COMUNICACIÓN SERIAL.
